%%%%%%%%%%%%%%%%%%%%%%%%%%%%%%%%%%%%%%%
% Deedy - One Page Two Column Resume
% LaTeX Template
% Version 1.1 (30/4/2014)
%
% Original author:
% Debarghya Das (http://debarghyadas.com)
%
% Original repository:
% https://github.com/deedydas/Deedy-Resume
%
% IMPORTANT: THIS TEMPLATE NEEDS TO BE COMPILED WITH XeLaTeX
%
% This template uses several fonts not included with Windows/Linux by
% default. If you get compilation errors saying a font is missing, find the line
% on which the font is used and either change it to a font included with your
% operating system or comment the line out to use the default font.
% 
%%%%%%%%%%%%%%%%%%%%%%%%%%%%%%%%%%%%%%
% 
% TODO:
% 1. Integrate biber/bibtex for article citation under publications.
% 2. Figure out a smoother way for the document to flow onto the next page.
% 3. Add styling information for a "Projects/Hacks" section.
% 4. Add location/address information
% 5. Merge OpenFont and MacFonts as a single sty with options.
% 
%%%%%%%%%%%%%%%%%%%%%%%%%%%%%%%%%%%%%%
%
% CHANGELOG:
% v1.1:
% 1. Fixed several compilation bugs with \renewcommand
% 2. Got Open-source fonts (Windows/Linux support)
% 3. Added Last Updated
% 4. Move Title styling into .sty
% 5. Commented .sty file.
%
%%%%%%%%%%%%%%%%%%%%%%%%%%%%%%%%%%%%%%%
%
% Known Issues:
% 1. Overflows onto second page if any column's contents are more than the
% vertical limit
% 2. Hacky space on the first bullet point on the second column.
%
%%%%%%%%%%%%%%%%%%%%%%%%%%%%%%%%%%%%%%

\documentclass[]{deedy-resume-openfont}


\begin{document}

%%%%%%%%%%%%%%%%%%%%%%%%%%%%%%%%%%%%%%
%
%     LAST UPDATED DATE
%
%%%%%%%%%%%%%%%%%%%%%%%%%%%%%%%%%%%%%%
\lastupdated

%%%%%%%%%%%%%%%%%%%%%%%%%%%%%%%%%%%%%%
%
%     TITLE NAME
%
%%%%%%%%%%%%%%%%%%%%%%%%%%%%%%%%%%%%%%


%\namesection{Amit}{Kumar}{ \urlstyle{same}\url{http://iamit.in} \\
%\href{mailto:dtu.amit@gmail.com}{dtu.amit@gmail.com} | +91 9811522423
%}

\namesection{Gaurav}{Dhingra}{ \urlstyle{same}\url{https://gxyd.github.io} \\
\href{mailto:gauravdhingra.gxyd@gmail.com}{gauravdhingra.gxyd@gmail.com} | +91 8791414504
}

%%%%%%%%%%%%%%%%%%%%%%%%%%%%%%%%%%%%%%
%
%     COLUMN ONE
%
%%%%%%%%%%%%%%%%%%%%%%%%%%%%%%%%%%%%%%

\begin{minipage}[t]{0.33\textwidth}

%%%%%%%%%%%%%%%%%%%%%%%%%%%%%%%%%%%%%%
%     EDUCATION
%%%%%%%%%%%%%%%%%%%%%%%%%%%%%%%%%%%%%%

\section{Education}

%\subsection{Delhi Tech. University}
%\descript{(Delhi College of Engineering)}
%\descript{Mathematics \& Computing Engineering}
%\location{2012 - 2016 | New Delhi \\ Aggregate: 61.40\%}
%\sectionsep

\subsection{Indian Institute of \\ Technology Roorkee}
\descript{Master of Science, Bachelor of Science in Applied Mathematics}
\location{2013 - 2018 (expected) \\ GPA: 7.188/10}
\sectionsep

% \descript{BS in Computer Science}
% \location{Expected May 2014 | Ithaca, NY}
% Conc. in Software Engineering \\
% College of Engineering \\
% Dean's List (All Semesters) \\
% \location{ Cum. GPA: 3.92 / 4.0 \\
% Major GPA: 3.94 / 4.0}
% \sectionsep

\section{Open Source}
    \textbullet{} \href{https://github.com/sympy/sympy}{\custombold{SymPy}} \textbullet{} \href{http://mpmath.org/}{\custombold{mpmath}} \textbullet{} \href{https://www.libreoffice.org}{\custombold{LibreOffice}}

%%%%%%%%%%%%%%%%%%%%%%%%%%%%%%%%%%%%%%
%     LINKS
%%%%%%%%%%%%%%%%%%%%%%%%%%%%%%%%%%%%%%

\section{Link}
Github:// \href{https://github.com/gxyd}{\custombold{gxyd}} \\
Web:// \href{https://gxyd.github.io}{\custombold{gxyd.github.io}} \\
Twitter://  \href{https://twitter.com/axyd0000}{\custombold{@axyd0000}} \\
\sectionsep

%%%%%%%%%%%%%%%%%%%%%%%%%%%%%%%%%%%%%%
%     COURSEWORK
%%%%%%%%%%%%%%%%%%%%%%%%%%%%%%%%%%%%%%

% \section{Coursework}
% \subsection{Graduate}
% Advanced Machine Learning \\
% Open Source Software Engineering \\
% Advanced Interactive Graphics \\
% Compilers + Practicum \\
% Cloud Computing \\
% \sectionsep

\subsection{Coursework}
Design \& Analysis of Algorithms \\
Graph Theory \\
Data Structures \\
Introduction to Linux {*}\\
Statistical Inference \\
Linear Algebra \\
Discrete Mathematics \\
Copyright {*}
\\
({*} are MOOCs)
\sectionsep

%%%%%%%%%%%%%%%%%%%%%%%%%%%%%%%%%%%%%%
%     SKILLS
%%%%%%%%%%%%%%%%%%%%%%%%%%%%%%%%%%%%%%

\section{Skills}
\subsection{Programming}
\location{Proficient:}
\textbullet{} Python \\
\location{Competent:}
\textbullet{} C \textbullet{} C++ \textbullet{} BASH \\
\location{Familiar:}
\textbullet{} JavaScript \textbullet{} MySQL \textbullet{} CSS \\
\subsection{Operating System}
\textbullet{} GNU/Linux \textbullet{} Windows
\subsection{Tools \& Framework}
\textbullet Vim \textbullet{} Git \textbullet{} Bootstrap
\sectionsep

%%%%%%%%%%%%%%%%%%%%%%%%%%%%%%%%%%%%%%
%     CONFERENCES
%%%%%%%%%%%%%%%%%%%%%%%%%%%%%%%%%%%%%%

\section{Talks}
\textbullet{} Lightning Talk "Why Python is good for mathematical computation", PyDelhi 2016 \\
\sectionsep

%%%%%%%%%%%%%%%%%%%%%%%%%%%%%%%%%%%%%%
%
%     COLUMN TWO
%
%%%%%%%%%%%%%%%%%%%%%%%%%%%%%%%%%%%%%%

\end{minipage}
\hfill
\begin{minipage}[t]{0.66\textwidth}

%%%%%%%%%%%%%%%%%%%%%%%%%%%%%%%%%%%%%%
%     EXPERIENCE
%%%%%%%%%%%%%%%%%%%%%%%%%%%%%%%%%%%%%%

\section{Experience}

\runsubsection{SymPy}
\descript{| Pull Request Manager }
\location{September 2017 - Present}
\vspace{\topsep} % Hacky fix for awkward extra vertical space
\vspace{\topsep}
\begin{tightemize}
\item SymPy is a popular python library for symbolic computation with more than 4000 stars on github.
\item Responsible to ensure that SymPy pull requests get reviewed quickly and help in SymPy release process.
\item A position funded by NumFOCUS.
\item Chosen for the position since of being one of the top contributors to SymPy.
\end{tightemize}
\sectionsep

\runsubsection{Google Summer of Code 2017}
\descript{| SymPy }
\location{May - July, 2017}
\vspace{\topsep} % Hacky fix for awkward extra vertical space
\begin{tightemize}
\item Worked on extending the computations using the Risch integration algorithm.
\item Implemented algorithm for parametric logarithmic derivative problem.
\item Trigonometric functions can now be integrated using the Risch algorithm.
\end{tightemize}
\sectionsep

\runsubsection{Google Summer of Code 2016}
\descript{| SymPy }
\location{April - Aug, 2016}
\vspace{\topsep} % Hacky fix for awkward extra vertical space
\begin{tightemize}
\item Created capability to do computation with Finite Groups and Finitely Presented Groups.
\item Implemented coset enumeration algorithm for finitely presented groups.
\item Reidemeister Schreier, low index subgroup algorithm for doing computation with subgroups and order of groups.
\end{tightemize}
\sectionsep

%\runsubsection{HackerEarth}
%\descript{| Software Engineering Intern }
%\location{R \& D Team | June 2016 - September 2016}
%\vspace{\topsep} % Hacky fix for awkward extra vertical space
%\begin{tightemize}
%\item Facilitated the collection of standard error logs.
%\item Wrote an Error Log Parser for C, C++, Python and Java.
%\item Wrote a Checker for comparing user output with expected output to give appropriate suggestion for frequently occurring cases. 
%\end{tightemize}
%\sectionsep

%\runsubsection{Google Summer of Code 2016}
%\descript{| SymPy }
%\location{Mentor for SymPy Project}
%\vspace{\topsep} % Hacky fix for awkward extra vertical space
%\begin{tightemize}\item Mentored two students for the development of solvers in SymPy. \href{https://summerofcode.withgoogle.com/projects/#6299625891823616}{[1]}, 
%\href{https://summerofcode.withgoogle.com/projects/#5440294841483264}{[2]}
%\end{tightemize}
%\sectionsep

%\runsubsection{Google Summer of Code 2015}
%\descript{| Python Software Foundation (SymPy) }
%\location{May 2015 – Aug 2015}
%\begin{tightemize}
%\item Worked on Solvers Module. Improved Mathematical robustness of new solvers module.
%\item Implemented Complex Sets: Representing infinite solutions in the argand plane;
%\item Linear System solver;
%\item Differential Calculus Methods.\end{tightemize}
%\sectionsep

% \runsubsection{SymEngine}
% \descript{| Open Source Contributor}
% \location{April 2015 | Couple of patches}
% \begin{tightemize}
% \item SymEngine is a fast symbolic manipulation library, written in C++. I contributed a couple
% of Patches (Pull Requests) Implemented Inverse Hyperbolic Secant Function.\end{tightemize}
% \sectionsep

%%%%%%%%%%%%%%%%%%%%%%%%%%%%%%%%%%%%%%
%     RESEARCH
%%%%%%%%%%%%%%%%%%%%%%%%%%%%%%%%%%%%%%

\section{PROJECTS}
\runsubsection{Series Convergence, Singularity and Accumulation Bounds in SymPy}
\descript{| Academic Project}
\location{January - April, 2016}
\vspace{\topsep}
Academic Project on implementation of sum and product convergence of series in SymPy, a computer algebra system.
Also implemented the Accumulation Bounds for assistance in computation of limits in SymPy.
\sectionsep

\runsubsection{Finding the level of Awareness and Acceptance of Ayurvedic products}
\descript{| Marketing Research}
\location{July - November 2015}
\vspace{\topsep} % Hacky fix for awkward extra vertical space
\begin{tightemize}
\item Conducted a study with reference to the disruption caused by Patanjali Products in FMCG Markets.
\item Tested multiple hypothesis based on analysis of the sample collected.
\item Compared the degree to which the respondents perceived ease of use and usefulness of online marketing with demographical questions.
\item Relationships among different factors were obtained using chi-square test.
\item Concluded that most people would find online marketing of ayurvedic products useful.
\end{tightemize}
\sectionsep

%\runsubsection{Meetler}
%\descript{| Mircrosoft Code.Fun.Do}
%
%\sectionsep

% \runsubsection{HashTweet}
% \descript{PHP application}
% Twitter API client in PHP to display ReTweets with a particular Hash Tag, using the Slim Framework.

% \sectionsep

%%%%%%%%%%%%%%%%%%%%%%%%%%%%%%%%%%%%%%
%     AWARDS
%%%%%%%%%%%%%%%%%%%%%%%%%%%%%%%%%%%%%%

% \section{Awards} 
% \begin{tabular}{rll}
% 2016	     & Qualified  & Google Code Jam, Qualification Round.\\
% 2015	     & HackerRank  & WorldCup semifinal with a team of two.\\
% 2014	     & Qualified  & Google Code Jam, Qualification Round.\\
% 2012     & National & Top 1\% out of 1.3 million Candidates in AIEEE.\\
% \end{tabular}
\sectionsep

%%%%%%%%%%%%%%%%%%%%%%%%%%%%%%%%%%%%%%
%     SPEAKING
%%%%%%%%%%%%%%%%%%%%%%%%%%%%%%%%%%%%%%

%%%%%%%%%%%%%%%%%%%%%%%%%%%%%%%%%%%%%%
%     PUBLICATIONS
%%%%%%%%%%%%%%%%%%%%%%%%%%%%%%%%%%%%%%

% \section{Publications} 
% \renewcommand\refname{\vskip -1.5cm} % Couldn't get this working from the .cls file
% \bibliographystyle{abbrv}
% \bibliography{publications}
% \nocite{*}
% [1]. SymPy Authors, "SymPy: Symbolic Computing in Python", 2016.

%%%%%%%%%%%%%%%%%%%%%%%%%%%%%%%%%%%%%%
%     SOCIETIES
%%%%%%%%%%%%%%%%%%%%%%%%%%%%%%%%%%%%%%

% \section{Societies} 

% \begin{tabular}{rll}
% 2014 	& top 12\%ile    & Tau Beta Pi Engineering Honor Society\\
% 2014   & National   & The Global Leadership and Education Forum (tGELF)\\
% 2012   &  National  & Golden Key International Honor Society\\
% 2012   &  National   & National Society of Collegiate Scholars\\
% \end{tabular}
% \sectionsep

\end{minipage} 
\end{document}  \documentclass[]{article}
